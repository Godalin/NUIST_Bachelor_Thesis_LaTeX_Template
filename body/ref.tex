\section{排版参考文献及引用}
\subsection{使用bibliography排版参考文献}
编写参考文献一章是个很无聊的工作,且工作量不小。使用“GB/T 7714—2015 BibTeX Style”排版参考文献可以省去模板用户研究参考文献排版规范的时间。尽管知网提供GB/T 7714—2015格式引文,但是知网提供的引文其实存在一个小问题:标点后没有空格,需要我们自行添加空格(知网外文文献提供的可供复制的引文做的更差,居然敢说是GB/T 7714—2015格式的……)。至于外文文献,引文基本上需要我们对照规范自行编辑。将参考文献排版工作交给程序,将我们从这个无聊且耗时的工作中解放出来。

模板用户需要编辑bibliography.bib文件,填写参考文献的各项属性,如title、author、year等,具体请参考\url{https://github.com/CTeX-org/gbt7714-bibtex-style#%E6%96%87%E7%8C%AE%E7%B1%BB%E5%9E%8B}。其实bib文件的生成工作也可以交给文献管理软件,进一步实现参考文献管理自动化。

bib文件示例:
{\color{green!50!black}
\begin{lstlisting}[breaklines=true,]
    @online{x1,
    title = {The Not So Short Introduction to LaTeX2e},
    year = {2021},
    author = {Tobias, O and Hubert, P and Irene, H and Elisabeth, S},
    url = {http://tug.ctan.org/info/lshort/english/lshort.pdf},
    urldate = {2021-06-05},
    langid = {english},
  }
  
  @book{x2,
    title = {LaTeX2e 及常用宏包使用指南},
    author = {李平},
    date = {2004},
    publisher = {{清华大学出版社}},
    location = {{北京}},
    langid = {中文;}
  }
\end{lstlisting}
}
示例中的x1、x2为参考文献的标识符,可以随意设定,在正文中使用\verb|\cite{x1}|命令即可实现文献交叉引用。

bib文件编辑完成后,用户需要依次进行下面四个操作:编译文档、在命令行环境执行bibtex命令、编译文档、编译文档。在Visual Studio Code中操作为:
\begin{enumerate}[1、]
    \item 点击\TeX 插件中的“Build LaTeX project”
    \item 在VSC的TERMINAL中执行“bibtex 文件名.aux”,如“bibtex NUIST\_thesis.aux”
    \item 点击\TeX 插件中的“Build LaTeX project”
    \item 点击\TeX 插件中的“Build LaTeX project”
\end{enumerate}
如果想了解原理,请参见\url{https://liam.page/2016/01/23/using-bibtex-to-generate-reference/}。用户也可以自行为\TeX 插件添加新的Recipe,一键完成编译。

若参考文献发生变更,或是编译发生错误,用户需要重新执行以上操作。

学校规定的参考文献排版规范有一点与国标不同:我校要求英文人名仅首字母大写。这要求模板用户在执行bibtex命令后,修改生成的bbl文件,将其中的英文人名改为符合我校要求的首字母大写、其余字母小写。

\subsection{(已弃用)thebibliography环境排版参考文献}
注释掉nuist.cls文件的第50行至第52行,并取消注释第55行至第57行,来启用thebibliography。

使用thebibliography环境来排版参考文献,代码如下:
{\color{green!50!black}
\begin{lstlisting}[breaklines=true,]
\begin{thebibliography}{99}\setlength{\itemsep}{-0.1mm}
\begin{spacing}{1.2}
\zihao{-5}
\bibitem{x1}The Not So Short Introduction to 
\LaTeX2e \ by Tobias Oetiker, Hubert Partl, Irene Hyna and Elisabeth Schlegl.
\bibitem{x0}李平.\LaTeX2e 及常用宏包使用指南[M].清华大学出版社,2004.
\bibitem{x3}罗振东,葛向阳.排版软件\LaTeX 简明手册[M].第二版.北京:电子工业出版社,2003.
\end{spacing}
\end{thebibliography}
\end{lstlisting}
}
引用文献条目时使用\verb|\ucite{}|命令,例如代码\verb|\ucite{x1}|和\verb|\ucite{x2}|就可以产生\cite{x1}和\cite{x2}上标。