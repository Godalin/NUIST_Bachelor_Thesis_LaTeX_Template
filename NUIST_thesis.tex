\documentclass{nuist}

% 参考文献文件 .bib
\addbibresource{bibliography.bib}

\begin{document}

%%%%%%%%%% 封面目录摘要 %%%%%%%%%%
% WARN: 请检查校徽是否为最新版,当前(2024 年)使用校徽为 2015 版;若不是最新版,请在 nuist_logo 文件夹中替换
% TIPS: 若标题、学院、专业名太长,可使用 \\ 进行换行,如 计算机学院 软件学院\\网络空间安全学院
% TIPS: 若不允许换行,请在 nuist.cls 文件中查找“封面”部分,并将 \parbox[b]{58mm} 中的 58 改为更大的数字
% TIPS: 2021 年论文格式要求指导老师不用加职称

\cover{
  南京信息工程大学本科生毕业论文\LaTeX{}模板\\
  Version ${2^0}_24$
}{
  路人甲
}{
  20200000666
}{
  \LaTeX{}学院
}{
  某专业
}{
  路人乙
}{
  二O二四\hspace{0.4em}年
  \hspace{0.4em}
  三\hspace{0.4em}月
  \hspace{0.4em}
  十八\hspace{0.4em}日
}

% 输出声明
\makestatement

% 使用罗马页码
\pagenumbering{Roman}
\mytableofcontents
\maketitleofchinese
{
南京信息工程大学本科生毕业论文\LaTeX{}模板V2022
\footnote{本模板制作时间:2014年5月,最后修订于2022年1月}
}{
Bruce~Y.P.~Lee\footnote{E-mail:
  \url{yupenglee119@gmail.com}}、
LiR\footnote{第二版修改者,E-mail:
  \url{stuliren@outlook.com}}、
John D\footnote{2021.6版修改者,E-mail:
  \url{mailto:work.temp.place@outlook.com}}、
B. Shen\footnote{2022版修改者,E-mail:
  \href{mailto:nj\_bwshen@outlook.com}{nj\_bwshen@outlook.com}}、
Godalin\footnote{2024版修订者,E-mail:
  \href{mailto:yly1228@foxmail.com}{yly1228@foxmail.com}}
}{
\LaTeX{}
}
\abstractofchinese{这是一份南京信息工程大学本科生毕业论文\LaTeX{}模板。
友善提醒:本文档是非官方版,属个人兴趣产物。}{模板;南信大;毕业论文;
}



\maketitleofenglish
{
\LaTeX\ Template for Undergraduate Thesis of Nanjing University of Information Science \& Technology
}{
Bruce Y.P. Lee、
LiR、John D、
B. Shen、
Godalin
}{
School of \LaTeX{}
}
\abstractofenglish
{
This is a \LaTeX{} template for the Undergraduate thesis of Nanjing University of Information Science \& Technology.
Caution: due to personal interest, not an official template.
}{
template;NUIST;thesis;
}


%%%%%%%%%% 正文 %%%%%%%%%%
% TIPS: 可以为每一章节在 body 文件夹内创建一个 ``.tex'' 文件,
% 并以下述方式引入,也可以直接写在本文件中(不推荐)

% 使用数字页码
\pagenumbering{arabic}

\input{body/intro.tex}
\input{body/nuistcommand.tex}
\input{body/math.tex}
\input{body/figs.tex}
\input{body/table.tex}
\input{body/code.tex}
\input{body/ref.tex}



%%%%%%%%%% 参考文献 %%%%%%%%%%
\printbibliography
\clearpage



% 2024 版模板致谢和附录次序交换



%%%%%%%%%% 致谢(可选) %%%%%%%%%%
% TIPS: \thanking 命令中包含了 \clearpage
\thanking{
  感谢春风之骀荡,感谢细雨之无声,感谢花枝之袅娜,感谢土地之坚忍。

  \vspace{5em}

  \color{red}\heartpar{
    鉴于您已经读到这里,噢,也可能是用鼠标拖到这里的,但这不是什么大不了的区别,重要的是您手指或者眼睛一定有一些疲倦了吧?这段文字能与您相遇,笔者心里已经满是感激,为了表达这样的心情,特奉上红心一颗,望看官笑纳!-- Lee贰零壹肆年伍月叁日于南京
  }
}



%%%%%%%%%% 附录(可选) %%%%%%%%%%
\input{body/appendix.tex}



\end{document}



% Local Variables:
% TeX-engine: xetex
% LaTeX-biblatex-use-Biber: t
% TeX-master: t
% End:
